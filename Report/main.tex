\listfiles
\documentclass[3p]{elsarticle}

\usepackage{lineno,hyperref}
\modulolinenumbers[5]

\journal{Applied Machine Learning Course}

%%%%%%%%%%%%%%%%%%%%%%%
%% Bibliography styles
%%%%%%%%%%%%%%%%%%%%%%%
%% To change the style, put a % in front of the second line of the current style and
%% remove the % from the second line of the style you would like to use.
%%%%%%%%%%%%%%%%%%%%%%%

% Numbered
% \bibliographystyle{model1-num-names}

%% Numbered without titles
% \bibliographystyle{model1a-num-names}

%% Harvard
% \bibliographystyle{model2-names}\biboptions{authoryear}

%% Vancouver numbered
% \usepackage{numcompress}\bibliographystyle{model3-num-names}

%% Vancouver name/year
% \usepackage{numcompress}\bibliographystyle{model4-names}\biboptions{authoryear}

%% APA style
% \bibliographystyle{model5-names}\biboptions{authoryear}

%% AMA style
% \usepackage{numcompress}\bibliographystyle{model6-num-names}

%% `Elsevier LaTeX' style, distributed in TeX Live 2019
\bibliographystyle{elsarticle-num}
% \usepackage{numcompress}\bibliographystyle{elsarticle-num-names}
% \bibliographystyle{elsarticle-harv}\biboptions{authoryear}
%%%%%%%%%%%%%%%%%%%%%%%

%%%%%%%%%%%%%%%%%%%%%%%%%%%%%%%%%%%%%%%%%%%%%%%%%%%%%%%%%%%%%%%
% DO NOT EDIT ANYTHING ABOVE THIS LINE
% EXCEPT IF YOU LIKE TO USE ADDITIONAL PACKAGES
%%%%%%%%%%%%%%%%%%%%%%%%%%%%%%%%%%%%%%%%%%%%%%%%%%%%%%%%%%%%%%%


\usepackage{comment}

\newcommand\blfootnote[1]{%
  \begingroup
  \renewcommand\thefootnote{}\footnote{#1}%
  \addtocounter{footnote}{-1}%
  \endgroup
}

\begin{document}

\begin{frontmatter}

%%%%%%%%% TITLE
%\title{Applied Machine Learning: Disease Symptom Prediction\\}
\title{Applied Machine Learning \\ \huge{\textbf{Disease Symptom Prediction}}}

%%%%%%%%% AUTHORS
\author{Louis Simon Spatscheck, Aikaterini Vasilopoulou, \\ 
Léandre Pablo Delphin Göblyös, Dominik Pastuszka Malek}
\let\thefootnote\relax\footnotetext{\textit{Email adresses:} author1@student.sdu.dk, author2@student.sdu.dk, author3@student.sdu.dk, dpast24@student.sdu.dk}

\begin{comment}
\author{Louis Simon Spatscheck}
\ead{author1@student.sdu.dk}
\author{Aikaterini Vasilopoulou}
\ead{author2@student.sdu.dk}
\author{Léandre Pablo Delphin Göblyös}
\ead{author3@student.sdu.dk}
\author{Dominik Pastuszka Malek}
\ead{dpast24@student.sdu.dk}
\end{comment}

%%%%%%%%% ABSTRACT
\begin{abstract}
The abstract for your project goes here. The length of the abstract should be between 200\-250 words. Tips for writing a good abstract can be found at  \url{https://www.ncbi.nlm.nih.gov/pmc/articles/PMC3136027/}. \\

%\noindent Link to dataset: 
%\begin{center}
%    \href{https://www.kaggle.com/datasets/itachi9604/disease-symptom-description-dataset}{Disease Symptom Prevention (Kaggle)}
%\end{center}

\noindent Link to \textit{GitHub} repository containing the dataset and all code developed for this project:
\begin{center}
    \href{https://github.com/domipm/AML-Disease-Symptom}{Applied Machine Learning - Disease Symptom Prevention (GitHub)}
\end{center}

\end{abstract}


\end{frontmatter}




% MAIN ARTICLE GOES BELOW
%%%%%%%%%%%%%%%%%%%%%%%%%%%%%%%%%%%%%%%%%%%%%%%%%%%%%%%%%%%%%%%


%%%%%%%%% BODY TEXT
\section{Introduction}

This template is based on the Elsevier \LaTeX\ template\footnote{\url{https://www.overleaf.com/latex/templates/elsevier-article-template-with-different-bibliography-styles/npwqmwvhvvrk}}. The information in this template is very minimal, and this file should serve you as a framework for writing your report. You may prefer to use a more collaboration-friendly tool while drafting the report with your class mates before you prepare the final report for submission. Remember that you should \textbf{submit both the report (PDF and \.tex files) and  the Python codes} you used for this project via \textbf{itslearning}. Also, \textbf{only one member per team} needs to submit the project material.


This is an example of a mathematical equation:
\begin{equation}
\[f(\mathbf{x}; \mathbf{w}) = \sum_{i=1}^{n} w_{ix}_i.\]
\end{equation}

This is a mathematical expression, $h(\mathbf{x}) = \hat{y}$ formatted in text. 

The project report should be \textbf{maximum 20 pages long (not counting references)} and should contain the sections that are already provided in this paper. Please check out the text in these sections for further information.


\subsection{Subsection}

You can use paragraphs or subsections to further structure your main sections. This is an example of a subsection.

\paragraph{This is a paragraph title} This is an example of a paragraph.
\begin{itemize}
\item Item number one.
\item Item number two.
\end{itemize}


\section{Related Work}

Related work should be discussed here. This is an example of a citation. To format the citations properly, put the corresponding references into the \textbf{mybibfile.bib} file. You can obtain BibTeX-formatted references for the \"bib\" file from Google Scholar  (\url{https://scholar.google.com}), for example, by clicking on the double-quote character under a citation and then selecting as shown in Figure

\begin{figure}[t]
\begin{center}
   \includegraphics[width=0.8\linewidth]{figures/google-scholar.pdf}
\end{center}
   \caption{Example illustrating how to get BibTeX references from
   Google Scholar.}\label{fig:google-scholar}
\end{figure}


Table~\ref{tab:some-table} shows an example for formatting a table.

\begin{table}
\begin{center}
\begin{tabular}{lc}

Method & Accuracy \\
\hline\hline
Method 1 & $70 \pm 3$ \% \\
Method 2 & $76 \pm 3$ \% \\

\end{tabular}
\end{center}\label{tab:some-table}
\caption{This is an example of a table.}
\end{table}

\section{Proposed Method}

Describe the method you are proposing, developing, or using. I.e., details of the algorithms may be included here. 

\section{Experiments}

Describe the experiments you performed. You may want to create separate subsections to further structure this section.

\subsection{Dataset}

Briefly describe your dataset in a separate subsection.


\subsection{Software}

Briefly list (and cite) software software you used.

\subsection{Hardware}

If relevant, list hardware resources you used.


\section{Results and Discussion}

Describe the results you obtained from the experiments and interpret them. Optionally, you could split Results and Discussion  into two separate sections.

\section{Conclusions}

Describe your conclusions here. If there are any future directions, you can describe them here, or you can create a new section for future directions.

\section{Acknowledgements}

List acknowledgements if any. For example, if someone provided you a dataset, or you used someone else's resources, this is a good place to acknowledge the help or support you received.

\section{Contributions}

Describe the contributions of each team member who worked on this project.



%%%%%%%%%%%%%%%%%%%%%%%%%%%%%%%%%%%%%%%%%%%%%%%%%%%%%%%%%%%%%%%
% DO NOT EDIT ANYTHING ABOVE THIS LINE
% EXCEPT IF YOU LIKE TO USE ADDITIONAL PACKAGES
%%%%%%%%%%%%%%%%%%%%%%%%%%%%%%%%%%%%%%%%%%%%%%%%%%%%%%%%%%%%%%%

\bibliography{mybibfile}

\end{document}